\documentclass{article}
\usepackage[a4paper, margin=1in]{geometry}
\usepackage[utf8]{inputenc}
\usepackage{eso-pic}
\usepackage{graphicx}
\usepackage{ngerman}
\usepackage{bbding}
\usepackage[weather]{ifsym}

\definecolor{ice-blue}{RGB}{162,210,223}

%\title{\bf Transformation Workshop}  % Supply information
\title{\textbf{Transformation Workshop}\\ZID Kt. Basel Stadt\\
\vspace{.5cm}\large \fontsize{40}{40}\color{ice-blue}\SnowflakeChevron }

\author{Bastian Bukatz}

\newcommand\BackgroundPic{%
\put(0,0){%
\parbox[b][\paperheight]{\paperwidth}{%
\vfill
\includegraphics[width=\paperwidth,height=\paperheight,keepaspectratio]{pictures/ipt_titelpage2.jpg}%
}}}


\begin{document}
\AddToShipoutPicture{\BackgroundPic}
\maketitle

\section*{Sinn und Zweck}
Ein Transformation Workshop macht Sinn, wenn ein Team oder eine Organisation vor einer grösseren Veränderung steht. Ein Transformation Workshop dient dann dem Team oder den Teams als Startschuss für den Veränderungprozess. Entsprechend stellt der Transformation Workshop ein wichtiges Element in der ersten Phase des IPT  Change Modell da, welches sich an dem 3-Phasen-Modell von Lewin orientiert.


\section*{Ziele}
Durch einen Transformation Workshop soll die Dringlickheit und die
Sinnhaftigkeit der anstehenden Veränderung vermittelt werden. Gleichzeitig
gibt der Workshop den Rahmen um die Mitarbeitenden in den anstehenden
Veränderungsprozesses einzubinden. Es wird ein ersten Wurf eines aus Sicht des Teams wünschenswertes Zielbild entwickelt.
\newline\textbf{Das Team sieht Bedarf für Veränderung und hat eine erste Vision, wo für sie die Reise hingehen soll.}


\section*{Workshop Übersicht}
Der Transformation Workshop gliedert sich in drei Abschnitte.
\begin{enumerate}
  \item \textbf{Eigenbild} Wie sieht sich das Team selbst. Wo sind sein Stärken und wo die Schwächen?
  \item \textbf{Fremdbild} Wichtige Stakeholder und Kunden geben Feedback zur Aussenwirkung des Teams. Wie wird das Team von Anderen wahrgenommen?
  \item \textbf{Zielbild} Anhand des Eigen- und Fremdbildes wird Potenzial für eine Verbesserung identifiziert und eine wünschenswertes Zielbild erarbeitet.
\end{enumerate}

\pagebreak

\section*{Workshop Agenda}

\noindent\textbf{Einstieg}
\begin{itemize}
  \item[] Begrüssung der Teilnehmer.
  \item[] Jeder hat eine Minute um eine Erfahrung aus dem Berufsleben und aus dem Privatleben vorzustellen, auf die er besonders stolz ist.
\end{itemize}

\subsection*{Selbstbild}
Das Team erarbeitet ein Selbstbild bezügich seiner Aussenwirkung. Was gelingt uns in der Zusammenarbeit mit unseren internen Kunden und Stakeholdern gut. Was gelingt uns nicht so gut?

\begin{description}
  \item [Selbsteinschätzung] \noindent Wo ist unsere Aussenwirkung besonders gut, wo sehen wir Verbesserungspotenzial? In Kleingruppen werden Beispiele auf Post-IT Zetteln gesammelt.
  \item [Konsolierung] \noindent Die Post-ITs werden konsolidiert und auf einem Flipchart gesammelt.
\end{description}

\subsection*{Fremdbild}
Die eingeladenen Gäste (interne Kunden und Stakeholder) stellen ihr Feedback dem Team vor. Wo leistet das Team aus ihrer Sicht besonders guten Outcome, wo sieht man Verbesserungspotenzial. Jeder Gast kann einen konkreten Wunsch an das Team äussern.

\begin{description}
  \item [Feedback] \noindent Jeder Gast stellt sein persönliches Feedback dem Team vor.
  \item [Konsolidierung] \noindent Das Feedback wird vom Moderator auf einem Flipchart konsolidiert.
  \item [Danke] \noindent Das Team bedankt sich für das Feedback und verabschiedet die Gäste.
\end{description}


\subsection*{Vision}
Resultat: Ein Plakat, welches die die Vision des Teams darstellt. Es enhält die wichtigsten Werte, auf denen die Veränderung aufbaut, wie wünschenswerten Ziel und wichtige Rahmenbedingungen, die es unbedingt zu beachten gibt.

\begin{description}
  \item[Bedarf] \noindent Das Team entscheided im Gremium, welches Feedback es für besonders wertvoll hält und woran man arbeiten will. Was will man undbedingt bewahren, wo will man sich verändern.
  \item[Zielbild] \noindent In Kleingruppen werden Ziele, Werte, Rahmenbedingungen auf Post-IT Zetteln gesammelt.
  \item[Konsolidierung] \noindent Konsolidierung der Resultate auf einem Flipchart im Gremium
\end{description}

\noindent\textbf{Abschluss}
\begin{itemize}
  \item[] Feedback der Teilnehmer zum Workshop und zum Ergebnis.
  \item[] Verabschiedung durch den Moderator.
\end{itemize}

\end{document}
